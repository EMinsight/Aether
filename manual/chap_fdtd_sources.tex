\section{Light sources}

\label{create_source_def}

\subsection{Common functions}

\subsubsection[location (three arguments)]{\lfc{location}(\lin{x1},\lin{y1},\lin{z1})}

Set the source location at (\lin{x1}, \lin{y1}, \lin{z1}). Those coordinates must be defined with respect to the computation geometry file, and this can be negative.

\subsubsection[location (six arguments)]{\lfc{location}(\lin{x1},\lin{x2},\lin{y1},\lin{y2},\lin{z1},\lin{z2})}

Set the source location between \lin{x1}, \lin{y1}, \lin{z1} (included) and \lin{x2}, \lin{y2}, \lin{z2} (excluded). Those coordinates must be defined with respect to the computation geometry file, and this can be negative.

\subsubsection[orientation]{\lfc{orientation}(\lsg{orientation})}

Defines the source orientation, that is \lsg{X}, \lsg{Y} or \lsg{Z}. This function is only ûsed by point sources.

\subsubsection[spectrum (one argument)]{\lfc{spectrum}(\lft{lambda})}

Defines the source wavelength as \lft{lambda}.

\subsubsection[spectrum (two arguments)]{\lfc{spectrum}(\lft{lambda\_min},\lft{lambda\_max})}

Defines the source spectrum as gaussian, between \lft{lambda\_min} and \lft{lambda\_max}.

\subsection{Source types}

\subsubsection{oscillator}

This source adds field to the simulation grid at each time step. This done through a operation like:

\begin{equation}
	E(t+1)=E(t)+sin(\omega t)
\end{equation}

As such, this souce is not exactly a radiating electric dipole, but just shows a similar behavior. Furthermore, field injection method can potentially create a static electric and magnetic field at the end of the simulation, and so this source must be used with care.
	
	This is a point source, and so its location is specified through the three arguments variant of the \lfc{location} function. The excited field component is set by the \lfc{orientation} function. The source has a wide spectrum, and so the two arguments variant of the \lfc{spectrum} function needs to be used.\\ Example:

\begin{lstlisting}
src=create_source("oscillator")
src:location(250,0,410)
src:orientation("Z")
src:spectrum(300e-9,1000e-9)

fdtd:register_source(src)
\end{lstlisting}

\subsubsection{Electric dipole}

Todo.

\subsubsection{Magnetic dipole}

Todo.