\chapter{The Materials Library}
\label{matsection}

\section{Constant materials}

Those are the most simple materials. They are characterized by a real, constant refractive index for every wavelength. They are first defined through the \lfc{set\_constant}() function, and the refractive index is then specified through either the \lfc{index\_infty}(\lft{index}) function, or the \lfc{epsilon\_infty}(\lft{epsilon}) function, $\varepsilon_r$ being the argument in the latter case.\\ Example:

\begin{lstlisting}
set_constant()
index_infty(1.5)
\end{lstlisting}

As explained in \ref{fdtd_fc_material}, in the case of a constant material the whole material script can also be replaced in the main script by a call to \lfc{const\_material}.

\section{Simple dispersive materials}

Those are the simplest models that are used to define materials like glasses in the spectral range where they are transparent. \textcolor{red}{Those materials are not FDTD compatible.}

\subsection{Cauchy}

\subsection{Sellmeier terms}

The Sellmeier formula is usually written as
\begin{equation}
	n^2=1+\sum_i \frac{B_i \lambda^2}{\lambda^2-C_i}
\end{equation}
but since $n^2=\varepsilon$, it can be understood as 
\begin{equation}
	\varepsilon=1+\sum_i \epsilon^\textrm{SM}_i ~~~~ ; ~~~~ \epsilon^\textrm{SM}_i=\frac{B_i \lambda^2}{\lambda^2-C_i}
\end{equation}
where each $\epsilon^\textrm{SM}_i$ resembles a Lorentz model.

Since $C_i$ obviously must have a dimension of m$^2$, it was decided it would be simpler for Aether to write $\epsilon^\textrm{SM}$ as
\begin{equation}
	\epsilon^\textrm{SM}_i=\frac{B_i \lambda^2}{\lambda^2-C_i^2}
\end{equation}
so that $C_i$ and $\lambda$ are homogeneous.

As such each Sellmeier term is added to the material dielectric model with the function \lfc{add\_sellmeier(\lft{B},\lft{C})}, where $\lft{C}$ is the square-root of what is usually found in material databases.

\section{Absorbing dispersive materials}

	Those materials are another kind of elementary material, mostly used to describe metals. They are implemented in the software through the recursive convolution method \cite{Luebbers:90}.
	
	First, the material is set as dielectric through the \lfc{set\_dielectric}() function. \lft{$\varepsilon_\infty$} is then defined thanks to \lfc{epsilon\_infty}(\lft{$\varepsilon_\infty$}). This value is real. The full dielectric model can be a combination of the elementary dielectric models, and is thus defined through successive calls of the \lfc{add\_drude}, \lfc{add\_lorentz} or \lfc{add\_crit\_points} functions. As the number of elementary dielectric models increases, so will the amount of needed computer memory, and the computation time.
	
	Finally, the \lfc{evaluate}(\lin{Nl},\lft{$\lambda_1$},\lft{$\lambda_2$}) function evualuates the dielectric model between \lft{$\lambda_1$} and \lft{$\lambda_2$} with \lin{Nl} sampling points. The result is written in a file called \lsgnq{mat\_eval.m}. It can be loaded in MatLab or GNU Octave.

\subsection{Drude model}

This basic model is very useful to describe metals in the infrared part of the spectrum. It is defined by
\begin{equation}
\varepsilon(\omega)=\varepsilon_\infty-\frac{\dsp \omega_D^2}{\dsp \omega^2+i\gamma\omega}
\end{equation}
and is turned into through \lfc{add\_drude}(\lft{$\omega_D$},\lft{$\gamma$}). This model is FDTD compatible.

\subsection{Lorentz model}

Another basic model useful for the description of metals, defined by
\begin{equation}
\varepsilon(\omega)=\varepsilon_\infty+A\frac{\dsp \Omega^2}{\dsp \Omega^2-\omega^2-i\Gamma\omega}
\end{equation}
it is defined through \lfc{add\_lorentz}(\lft{$A$},\lft{$\Omega$},\lft{$\Gamma$}) This model is FDTD compatible.

\subsection{Critical points model}

This model was initially introduced to described the permittivity of gold in the blue part of the spectrum \cite{Etchegoin:06}, and was then adapted to FDTD by \cite{Vial:07}. Its dielectric function is
	
\begin{equation}
\varepsilon(\omega)=\varepsilon_\infty+A\Omega\left(\frac{\dsp e^{\dsp i\varphi}}{\dsp \Omega-\omega-i\Gamma}+\frac{\dsp e^{\dsp -i\varphi}}{\dsp \Omega+\omega+i\Gamma}\right)
\end{equation}
and this model is used through \lfc{add\_crit\_point}(\lft{$A$},\lft{$\Omega$},\lft{$\varphi$},\lft{$\Gamma$}). This model is FDTD compatible.

\subsection{Full example}

The following example defines a gold dielectric model from 400 to 1000 nm. The permittivity first is set to 1.03, and a drude model and two critical points models are then added. The validity range is then set based on the original data fit.

\begin{lstlisting}
set_dielectric()

epsilon_infty(1.03)
add_drude(1.3064e16,1.1274e14)
add_crit_point(0.86822,4.0812e15,-0.60756,7.3277e14)
add_crit_point(1.3700,6.4269e15,-0.087341,6.7371e14)

validity_range(400e-9,1000e-9)
\end{lstlisting}

\section{Data tables based materials}

Sometimes, the dielectric model of a particular material is not known, and only tabulated data is available. It is possible to use that data to define a material within Aether.

This data is to be defined through three Lua tables, that will then be fed to a function:
\begin{itemize}
	\item the wavelength, in meters
	\item the real part of the input
	\item the imaginary part of the input
\end{itemize}
Those three tables must contain the same number of points. Then, they are passed as arguments to either
\begin{enumerate}
	\item \lfc{add\_data\_epsilon}(\lud{lambda},\lud{real},\lud{imag}): takes the permittivity as an input
	\item \lfc{add\_data\_index}(\lud{lambda},\lud{real},\lud{imag}): takes the refractive index as an input, which is internally turned into a permittivity  before being memorized by the material
\end{enumerate}
Example:
\begin{lstlisting}
lambda={400e-9,500e-9,600e-9}
nr={2.3,2.04,1.96}
ni={2e-2,1e-2,1e-2}

add_data_index(lambda,nr,ni)
\end{lstlisting}

Of course, it is impossible to provide data for every wavelength. As such, \textbf{the permittivity between the data points is reconstructed through cubic splines interpolation}.