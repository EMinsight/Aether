\section{Sensors}

\label{create_sensor_def}

\subsection{Common functions}

\subsubsection[location]{\lfc{location}(\lin{x1},\lin{x2},\lin{y1},\lin{y2},\lin{z1},\lin{z2})}

Sets the sensor location between \lin{x1}, \lin{y1}, \lin{z1} (inclus) and \lin{x2}, \lin{y2}, \lin{z2} (exclus). Those coordinates must be defined with respect to the computation geometry file, and this can be negative.

\subsubsection[name]{\lfc{name}(\lsg{name})}

Gives the sensor the name \lsg{name}, which will serve as a prefix to the names of the files written at the end of the computation.

\subsubsection[orientation]{\lfc{orientation}(\lsg{orientation})}

This defines the sensor orientation (its normal vector), through the \lsg{orientation} variable. Its value can be \lsg{X}, \lsg{Y}, \lsg{Z}, \lsg{-X}, \lsg{-Y} or \lsg{-Z}. This function has two main goals:
\begin{itemize}
	\item setting a direction onto which the Poynting vector will be projected, for the sensors that need it.
	\item specifying the exact location of the sensor, which was previously defined through the \lfc{location} function. For instance, if the orientation is \lsg{X} or \lsg{-X} the \lin{x2} variable will be ignored. Similarly, \lsg{Y} will affect \lin{y2} and \lsg{Z} will affect \lin{z2}.
\end{itemize}

\subsubsection[spectrum (one argument)]{\lfc{spectrum}(\lft{lambda})}

Sets the sensor analysis wavelength \lft{lambda}.

\subsubsection[spectrum (three arguments)]{\lfc{spectrum}(\lft{lambda\_min},\lft{lambda\_max},\lin{Nl})}

Sets the sensor analysis spectrum between \lft{lambda\_min} and \lft{lambda\_max}, with \lin{Nl} sampling points.

\subsection{Sensor types}

\subsubsection{fieldmap}

This sensors is dedicated to creating 2D fieldmaps. The fieldmap is computed for a single wavelength, specified through the single argument variant of \lfc{spectrum}.
	
At the end of the simulation, several files are created:
\begin{itemize}
	\item \lsgnq{name\_fieldmap.bmp}: a quick fieldmap image using a non-linear normalization
	\item \lsgnq{name.m}: a MatLab script that will display the fieldmap
	\item \lsgnq{name\_Ex\_raw}: a file containing the sensor data for the $x$ component of the field
	\item \lsgnq{name\_Ey\_raw}: a file containing the sensor data for the $x$ component of the field
	\item \lsgnq{name\_Ez\_raw}: a file containing the sensor data for the $x$ component of the field
\end{itemize}
Example:
\begin{lstlisting}
map=create_sensor("fieldmap")
map:spectrum(405e-9)
map:name("map")
map:orientation("Y")
map:location(0,500,0,1,0,500)

fdtd:register_sensor(map)
\end{lstlisting}

\subsubsection{fieldblock}

\subsubsection{planar\_spectral\_poynting}

\fwarn
This sensor integrates the Poynting vector flux across a planar surface.
